%%%%%%%%%%%%%%%%%%%%%%%%%%%%%%%%%%%%%%%%%
% baposter Landscape Poster
% LaTeX Template
% Version 1.0 (11/06/13)
%
% baposter Class Created by:
% Brian Amberg (baposter@brian-amberg.de)
%
% This template has been downloaded from:
% http://www.LaTeXTemplates.com
%
% License:
% CC BY-NC-SA 3.0 (http://creativecommons.org/licenses/by-nc-sa/3.0/)
%
%%%%%%%%%%%%%%%%%%%%%%%%%%%%%%%%%%%%%%%%%

%----------------------------------------------------------------------------------------
%	PACKAGES AND OTHER DOCUMENT CONFIGURATIONS
%----------------------------------------------------------------------------------------

\documentclass[landscape,a0paper,fontscale=0.35]{baposter} % Adjust the font scale/size here

\usepackage{graphicx} % Required for including images
\graphicspath{{figures/}} % Directory in which figures are stored
\usepackage[ngerman]{babel}
\usepackage{amsmath} % For typesetting math
\usepackage{amssymb} % Adds new symbols to be used in math mode
\usepackage[utf8]{inputenc}
\usepackage{booktabs} % Top and bottom rules for tables
\usepackage{enumitem} % Used to reduce itemize/enumerate spacing
\usepackage{helvet} % Use the Palatino font
\usepackage{url}
\renewcommand{\familydefault}{\sfdefault}
\usepackage[font=small,labelfont=bf]{caption} % Required for specifying captions to tables and figures

\usepackage{multicol} % Required for multiple columns
\setlength{\columnsep}{1.5em} % Slightly increase the space between columns
\setlength{\columnseprule}{0mm} % No horizontal rule between columns

\usepackage{tikz} % Required for flow chart
\usetikzlibrary{shapes,arrows} % Tikz libraries required for the flow chart in the template

\newcommand{\compresslist}{ % Define a command to reduce spacing within itemize/enumerate environments, this is used right after \begin{itemize} or \begin{enumerate}
\setlength{\itemsep}{1pt}
\setlength{\parskip}{0pt}
\setlength{\parsep}{0pt}
}

\definecolor{lightblue}{rgb}{0.145,0.6666,1} % Defines the color used for content box headers

\begin{document}

\begin{poster}
{
headerborder=closed, % Adds a border around the header of content boxes
colspacing=1em, % Column spacing
bgColorOne=white, % Background color for the gradient on the left side of the poster
bgColorTwo=white, % Background color for the gradient on the right side of the poster
borderColor=black, % Border color
headerColorOne=gray, % Background color for the header in the content boxes (left side)
headerColorTwo=gray, % Background color for the header in the content boxes (right side)
headerFontColor=white, % Text color for the header text in the content boxes
boxColorOne=white, % Background color of the content boxes
textborder=roundedleft, % Format of the border around content boxes, can be: none, bars, coils, triangles, rectangle, rounded, roundedsmall, roundedright or faded
eyecatcher=true, % Set to false for ignoring the left logo in the title and move the title left
headerheight=0.1\textheight, % Height of the header
headershape=rectangle, % Specify the rounded corner in the content box headers, can be: rectangle, small-rounded, roundedright, roundedleft or rounded
headerfont=\Large\bf\textsc, % Large, bold and sans serif font in the headers of content boxes
%textfont={\setlength{\parindent}{1.5em}}, % Uncomment for paragraph indentation
linewidth=2pt % Width of the border lines around content boxes
}
%----------------------------------------------------------------------------------------
%	TITLE SECTION 
%----------------------------------------------------------------------------------------
%
{\includegraphics[height=6em]{favicon.png}} % First university/lab logo on the left
{\fontsize{32}{32} \textbf{\textsc{Implementation of an Invitation Management System}}\vspace{0.5em}} % Poster title 
{\textsc{\{ Steffen Walter, Sidney Kuyateh, Lukas Schnaithmann, Lukas Priester \} \hspace{12pt} Invitations-Factory}} % Author names and institution
{\includegraphics[height=6em]{favicon.png}} % Second university/lab logo on the right

%----------------------------------------------------------------------------------------
%	OBJECTIVES
%----------------------------------------------------------------------------------------

\headerbox{Einleitung}{name=objectives,column=0,row=0}{

Im Rahmen eines Modules sollte ein Programm entworfen und ausgearbeitet werden. Für das entwickeln eines Programms gibt es mehrere Anforderung die zu beachten sind.

\begin{enumerate}\compresslist
\item Es soll nach Scrum entwickelt werden
\item Das Programm soll am Ende verkaufsfähig sein
\item Unit-test sind zu implementieren
\item Umsetzung der durch die Referate vermittelten Inhalte\\
\end{enumerate}

\vspace{0.3em} % When there are two boxes, some whitespace may need to be added if the one on the right has more content
}


%----------------------------------------------------------------------------------------
%	RESULTS 1
%----------------------------------------------------------------------------------------

\headerbox{Die Idee}{name=results,column=1,span=2,row=0,}
{
	\begin{multicols}{2}
	Die Idee entstand im Rahmen einer Teamsitzung. Ziel war es den Menschen etwas zu bieten was gebraucht wird.\\
	Das Programm dient dem Anwender dafür, seine Einladungen schneller und Effizienter zu verschicken.
	So entstand die Idee zu einer Webanwendung zur Verwaltung von Einladungen. Die Abbildung \ref{Ideenfindung} zeigt ein Ausschnitt des ersten Entwurfs. Nachlesen kann man diesen auch hier:\\ \url{https://github.com/stuttgart-dhbw/com.dhbw.team3/wiki/Ideenfindung}\\
	\begin{center}
		\includegraphics[width=0.7\linewidth]{Ideenfindung_Kleiner.PNG}
		\captionof{figure}{Ideenfindung im Wiki auf GitHub}
		\label{Ideenfindung}
	\end{center}
	
	\end{multicols}
}
%
%\begin{multicols}{2}
%\vspace{1em}
%\begin{center}
%\includegraphics[width=0.8\linewidth]{placeholder}
%\captionof{figure}{Figure caption}
%\end{center}
%
%Aliquam auctor, metus id ultrices porta, risus enim cursus sapien, quis iaculis sapien tortor sed odio. Mauris ante orci, euismod vitae tincidunt eu, porta ut neque. Aenean sapien est, viverra vel lacinia nec, venenatis eu nulla. Maecenas ut nunc nibh, et tempus libero. Aenean vitae risus ante. Pellentesque condimentum dui. Etiam sagittis purus non tellus tempor volutpat. Donec et dui non massa tristique adipiscing.
%\end{multicols}
%
%%------------------------------------------------
%
%\begin{multicols}{2}
%\vspace{1em}
%Sed fringilla tempus hendrerit. Vestibulum ante ipsum primis in faucibus orci luctus et ultrices posuere cubilia Curae; Etiam ut elit sit amet metus lobortis consequat sit amet in libero. Lorem ipsum dolor sit amet, consectetur adipiscing elit. Phasellus vel sem magna. Nunc at convallis urna. isus ante. Pellentesque condimentum dui. Etiam sagittis purus non tellus tempor volutpat. Donec et dui non massa tristique adipiscing. Quisque vestibulum eros eu.
%
%\begin{center}
%\includegraphics[width=0.8\linewidth]{placeholder}
%\captionof{figure}{Figure caption}
%\end{center}
%
%\end{multicols}
%}



%----------------------------------------------------------------------------------------
%	RESULTS 2
%----------------------------------------------------------------------------------------


%Donec faucibus purus at tortor egestas eu fermentum dolor facilisis. Maecenas tempor dui eu neque fringilla rutrum. Mauris \emph{lobortis} nisl accumsan.

%\begin{center}
%\begin{tabular}{l l l}
%\toprule
%\textbf{Treatments} & \textbf{Response 1} & \textbf{Response 2}\\
%\midrule
%Treatment 1 & 0.0003262 & 0.562 \\
%Treatment 2 & 0.0015681 & 0.910 \\
%Treatment 3 & 0.0009271 & 0.296 \\
%\bottomrule
%\end{tabular}
%\captionof{table}{Table caption}
%\end{center}

%Nulla ut porttitor enim. Suspendisse venenatis dui eget eros gravida tempor. Mauris feugiat elit et augue placerat ultrices. Morbi accumsan enim nec tortor consectetur non commodo.

%\begin{center}
%\begin{tabular}{l l l}
%\toprule
%\textbf{Treatments} & \textbf{Response 1} & \textbf{Response 2}\\
%\midrule
%Treatment 1 & 0.0003262 & 0.562 \\
%Treatment 2 & 0.0015681 & 0.910 \\
%Treatment 3 & 0.0009271 & 0.296 \\
%\bottomrule
%\end{tabular}
%\captionof{table}{Table caption}
%\end{center}
%}



%----------------------------------------------------------------------------------------
%	Sicherheitserwägung
%----------------------------------------------------------------------------------------

\headerbox{Sicherheitserwägungen}{name=introduction,column=3,row=0,span=1}{

Folgende Maßnahmen wurden ergriffen um die Sicherheit des Anwendungsservers, des Webservers, des Mailservers und der Datenbank für die Benutzer und Administratoren auf eine möglichst hohe Ebene zu heben.\\ 
Es wurde ein NGINX Server Installiert welcher als Erster Kontaktpunkt für jegliche http Anfragen dient und diese ggf. weiterleitet. Er ist derart konfiguriert, dass er den Ansprüchen von Mozillas Observatory mehr als zu gerecht wird. Hierfür wurden sehr strenge Sicherheitsvorkehrungen implementiert.\\ Genaueres kann hier nachgelesen werden: \url{https://observatory.mozilla.org/analyze/invitation-factory.tk}
\begin{center}
	\includegraphics[width=0.5\linewidth]{Observatory.PNG}
	\captionof{figure}{Domain Überprüfung}
\end{center}
Als Mailserver kommt Postfix zum Einsatz. Neben einer möglichst sicheren Konfiguration wurde zusätzlich ein Augenmerk darauf gelegt, dass E-Mails welche von diesem Server versendet werden, bei anderen Anbietern nicht als Spam eingeordnet werden. Um dies zu erreichen musste dafür gesorgt werden, dass die Serveranwendung den Mails entsprechende Header zufügt, dass die Postfix Konfiguration entsprechend eingstellt wurde und es mussten entsprechende DNS Einträge gemacht werden. Zum Test der Konfigurationen wurde ein Werkzeug namens Mail-Tester verwendet bei dem der Postfix Server mit Bestnote abschießt. Das Ergebnis kann hier eingesehen werden: \url{https://www.mail-tester.com/test-foxmj}.
\begin{center}
	\includegraphics[width=0.5\linewidth]{E-Mail_test.PNG}
	\captionof{figure}{Test um verschickte E-Mail zu überprüfen}
\end{center}
Die Webanwendung hört nur auf einen lokalen Port und wird über den NGINX Server der Außenwelt zur Verfügung gestellt.\\
Die Transportsicherheit wird also durch den NGINX Server gewährleistet welcher mit einem Wildcard Let's Encrypt Zertifikat abgesichert ist.\\
Benutzerpasswörter werden mit dem bcrypt Algorithmus gehasht und nicht im Klartext in der Datenbank abgelegt.\\
Die Anmeldung auf dem Server ist ausschließlich mit über SSH und RSA pubic/private Keys möglich. 
\\

}


%----------------------------------------------------------------------------------------
%	REFERENCES
%----------------------------------------------------------------------------------------
%
%\headerbox{References}{name=references,column=0,above=bottom}{
%
%\renewcommand{\section}[2]{\vskip 0.05em} % Get rid of the default "References" section title
%\nocite{*} % Insert publications even if they are not cited in the poster
%\small{ % Reduce the font size in this block
%\bibliographystyle{unsrt}
%\bibliography{sample} % Use sample.bib as the bibliography file
%}}

%----------------------------------------------------------------------------------------
%	FUTURE RESEARCH
%----------------------------------------------------------------------------------------

%\headerbox{Future Research}{name=futureresearch,column=1,span=2,aligned=references,above=bottom}{ % This block is as tall as the references block
%
%\begin{multicols}{2}
%Integer sed lectus vel mauris euismod suscipit. Praesent a est a est ultricies pellentesque. Donec tincidunt, nunc in feugiat varius, lectus lectus auctor lorem, egestas molestie risus erat ut nibh.
%
%Maecenas viverra ligula a risus blandit vel tincidunt est adipiscing. Suspendisse mollis iaculis sem, in \emph{imperdiet} orci porta vitae. Quisque id dui sed ante sollicitudin sagittis.
%\end{multicols}
%}

%----------------------------------------------------------------------------------------
%	CONTACT INFORMATION
%----------------------------------------------------------------------------------------

%\headerbox{Contact Information}{name=contact,column=3,aligned=references,above=bottom}{ % This block is as tall as the references block
%
%\begin{description}\compresslist
%\item[Web] www.university.edu/smithlab
%\item[Email] john@smith.com
%\item[Phone] +1 (000) 111 1111
%\end{description}
%}



%----------------------------------------------------------------------------------------
%	Projektverlauf
%----------------------------------------------------------------------------------------

\headerbox{Projektverlauf }{name=method,column=0,below=results,span=2,aligned=references}{ % This block's bottom aligns with the bottom of the conclusion block

Bei diesem Projekt wurde viel Wert auf die agile Projektentwicklung gelegt. Bei diesem System handelt es sich um Scrum.
\begin{center}
	\includegraphics[width=0.7\linewidth]{Scrum_Board_GitHub2.PNG}
	\captionof{figure}{Scrumboard auf GitHub}
\end{center}

Das Projekt beläuft sich auf einen Zeitraum von 2 Monaten. Dieser Zeitraum wurde in 3 Sprints eingeteilt. Dabei wurde vom Product Owner und dem Entwicklerteam für jeden Sprint User Stories, also Aufgaben ausgearbeitet und verteilt.\\
Das nachfolgende Diagramm soll den Zeitraum mit den Sprints und ihreren jeweiligen Aufgaben zeigen.
\begin{center}
	\includegraphics[width=0.7\linewidth]{GanttDiagramm.PNG}
	\captionof{figure}{GanttDiagramm}
\end{center}

}

%----------------------------------------------------------------------------------------
%	Methode
%----------------------------------------------------------------------------------------


\headerbox{Methode}{name=results2,column=2,span=1,aligned=method}{ 
Zur Realisierung des Projektes kommt das Framwork \"gobuffalo\" zum Einsatz. Dabei handelt es sich um eine Umgebung, die es sich zum Ziel gemacht hat, eine Webapplikation unter verwendung der Programmiersprache GO möglichst schnell und einfach zur Enwickeln zu können. Zum Einsatz kommen neben GO noch Technologien wie JavaScript, HTML5, CSS und weitere. Die Kernfunktionen von gobuffalo werden in Abbildung \ref{features} dargestellt.
\begin{center}
	\includegraphics[width=0.6\linewidth]{buffalo_features.png}
	\captionof{figure}{Kernfunktionen von gobuffalo}
	\label{features}
\end{center}

}

%----------------------------------------------------------------------------------------
%	CONCLUSION
%----------------------------------------------------------------------------------------

\headerbox{Ergebnis}{name=conclusion,column=0,span=3,row=1,below=method,above=bottom}
{
	\begin{multicols}{3}
		Das Ergebnis dieses Projekts ist mehr als zufriedenstellen. Unsere Webseite \url{invitation-factory.tk} bietet dem Anwender genau das was sie wollen. Es bietet ihnen die Möglichkeit Freunde,Familie oder Kollegen eine Einladung zu verschicken ohne großen Aufwand. Nach einer erfolgreichen Registrierung ist es dem Anwender möglich eine Einladung mit dem gleichen Textinhalt an mehrere Personen zeitgleich zu verschicken.\\
<<<<<<< HEAD
		\\
		Ein Extra das uns von anderen unterscheiden soll ist die Statusüberprüfung. Der Anwender hat dadurch die Möglichkeit den Status seiner Gäste einzusehen. Wir bieten ihm also eine separate Seite an, wo er schauen kann welcher Gast zu-, abgesagt oder sich noch nicht entschieden hat.\\
		Die Webseite ist also Anwenderfreundlich und vor allem auch effizient.
=======
		
		Ein Extra das uns von anderen unterscheiden soll ist die Statusüberprüfung. Der Anwender hat dadurch die Möglichkeit den Status seiner Gäste einzusehen. Wir bieten ihm also eine separate Seite an, wo er schauen kann welcher Gast zu-,abgesagt oder sich noch nicht entschieden hat.\\
		Die Webseite ist also Anwenderfreundlich und vor allem auch effizient. In Abbildung \ref{page_sample} wird exemplarisch die Erstellung einer Einladung dargestellt.
>>>>>>> 4057ce5ad87623be5f5e65e56c9f16b50798511f
		\begin{center}
					\includegraphics[width=0.6\linewidth]{New_Invitation_Seite.PNG}
			\captionof{figure}{Webseite für neue Einladungen}
			\label{page_sample}
		\end{center}

	\end{multicols}
	
	
	
}
%
%\begin{multicols}{2}
%
%\tikzstyle{decision} = [diamond, draw, fill=blue!20, text width=4.5em, text badly centered, node distance=2cm, inner sep=0pt]
%\tikzstyle{block} = [rectangle, draw, fill=blue!20, text width=5em, text centered, rounded corners, minimum height=4em]
%\tikzstyle{line} = [draw, -latex']
%\tikzstyle{cloud} = [draw, ellipse, fill=red!20, node distance=3cm, minimum height=2em]
%
%\begin{tikzpicture}[node distance = 2cm, auto]
%\node [block] (init) {Initialize Model};
%\node [cloud, left of=init] (Start) {Start};
%\node [cloud, right of=init] (Start2) {Start Two};
%\node [block, below of=init] (init2) {Initialize Two};
%\node [decision, below of=init2] (End) {End};
%\path [line] (init) -- (init2);
%\path [line] (init2) -- (End);
%\path [line, dashed] (Start) -- (init);
%\path [line, dashed] (Start2) -- (init);
%\path [line, dashed] (Start2) |- (init2);
%\end{tikzpicture}
%
%%------------------------------------------------
%
%\begin{itemize}\compresslist
%\item Pellentesque eget orci eros. Fusce ultricies, tellus et pellentesque fringilla, ante massa luctus libero, quis tristique purus urna nec nibh. Phasellus fermentum rutrum elementum. Nam quis justo lectus.
%\item Vestibulum sem ante, hendrerit a gravida ac, blandit quis magna.
%\item Donec sem metus, facilisis at condimentum eget, vehicula ut massa. Morbi consequat, diam sed convallis tincidunt, arcu nunc.
%\item Nunc at convallis urna. isus ante. Pellentesque condimentum dui. Etiam sagittis purus non tellus tempor volutpat. Donec et dui non massa tristique adipiscing.
%\end{itemize}
%
%\end{multicols}
%}


%----------------------------------------------------------------------------------------

\end{poster}

\end{document}